% Created 2019-06-01 Sat 12:56
% Intended LaTeX compiler: pdflatex
\documentclass[11pt]{article}
\usepackage[T1]{fontenc}
\usepackage{graphicx}
\usepackage{grffile}
\usepackage{longtable}
\usepackage{wrapfig}
\usepackage{rotating}
\usepackage[normalem]{ulem}
\usepackage{amsmath}
\usepackage{textcomp}
\usepackage{amssymb}
\usepackage{capt-of}
\usepackage{hyperref}
\usepackage{geometry}
\usepackage{fancyhdr}
\usepackage[Lenny]{fncychap}
\usepackage[figuresright]{rotating}
\usepackage{capt-of}
\usepackage{amssymb}
\usepackage[normalem]{ulem}
\usepackage{wrapfig}
\usepackage{grffile}
\usepackage{booktabs}
\usepackage{tabularx}
\usepackage{amsmath}
\usepackage{textcomp}
\usepackage{longtable}
\usepackage{float}
\usepackage{xunicode}
\usepackage{indentfirst}
\usepackage{xcolor}
\usepackage{fontspec}
\usepackage{tikz}
\usepackage{ctex}
\usepackage{minted}
\author{stardiviner}
\date{\today}
\title{简历}
\hypersetup{
 pdfauthor={stardiviner},
 pdftitle={简历},
 pdfkeywords={},
 pdfsubject={},
 pdfcreator={Emacs 27.0.50 (Org mode 9.2.1)}, 
 pdflang={English}}
\begin{document}

\maketitle

\section{自我介绍}
\label{sec:org9001276}

\begin{figure}[htbp]
\centering
\includegraphics[width=2.0in]{data/images/me_picture 23.jpg}
\caption{那时23岁的我}
\end{figure}


从 \textit{[2011-10-10 Mon] } 年以来,我一直使用 Linux 系统作为我的日常系统,完全取代
Windows系统。使用 Clojure 和 Python, Elisp 编程语言,熟悉 Web,App 爬虫。平常使
用 Git 管理代码,经常在GitHub上出没。用 Docker, Vagrant 设置开发环境。会写一点
Shell 脚本。求知欲强,自我学习驱动,善于解决问题。工作认真,性格平和亲近,擅长和
朋友聊天。

喜欢电脑技术,喜欢编程,喜欢互联网技术。喜欢看哲学书,电影,滑板,和史诗音乐。性
格随和,思想开放乐观,接受多元化文化。

\noindent\rule{\textwidth}{0.8pt}

\section{职业经历}
\label{sec:org1e942f8}

2011 年大学三年级辍学后:

\begin{itemize}
\item 在家自学过一段时间
\item 第一份工作是做的绿城房地产的销售
\item 之后必胜客的服务员,再做了一段时间的经理预备
\item 后来又做过一段时间的机器人编程的幼教老师
\item 给小公司做过IT设备维修支持,提供常见的IT维修服务。
\item 在一家网络公司“尚游网络科技有限公司”担任过推广,半个程序员运维工作。
\end{itemize}

\noindent\rule{\textwidth}{0.8pt}

\section{编程项目经历}
\label{sec:org162a77b}

\subsection{爬虫项目}
\label{sec:org3b64410}

\subsubsection{新旧网站的内容对比程序}
\label{sec:org037040c}
项目地址: \url{https://github.com/stardiviner/website-compare}

README有详细的记录。

其中爬取页面部分都是简单的,但是真正难的部分是比较部分。而且要求100\%的完整性比较
(包括文本内容,网页上的多媒体附件,以及导航栏的改动等等)。

\subsubsection{Slack聊天记录的爬虫}
\label{sec:org8569a54}
项目地址: \url{https://github.com/stardiviner/clojurians-slack-log}

README有详细的记录。

打算在其中加入Clojure的并发和并行支持。作为实验性质的。

\subsubsection{关于爬虫的项目经验}
\label{sec:orgafe34dc}

由于我还没有在互联网公司真正做过大的爬虫项目,所以算不上有足够经验。但是希望能给
我一个实习生的机会。

\subsection{Emacs / Org Mode}
\label{sec:org463a336}

\begin{itemize}
\item 给Emacs写过一些小插件
\item 给Org Mode提交过一些PR,特别是Org Babel Literate Programming 方面的支持,以及
ob-clojure 对 CIDER (Emacs 上 的 Clojure IDE) 的支持。
\end{itemize}

\subsection{小工具}
\label{sec:org12f5dea}

\subsubsection{Kindle笔记导出工具}
\label{sec:org27589e0}

写过一个工具支持从Kindle的笔记记录导出到Emacs Org Mode 格式。

\noindent\rule{\textwidth}{0.8pt}

\section{很多人都会问的关于我大学辍学这个选择}
\label{sec:org2072a8a}

我在安徽工程大学读大三下半年的时候辍学了,大学时学习一般,觉得学习很枯燥,以前念
书,是因为身边的人都在念书,可以这么说,以前念书是为了他人而念书,但是当我在大学
里想,我为什么念书?想的多了,人就困惑起来,和很多有过类似经历的人一样,在迷茫的
时期,经历过内心的挣扎,和煎熬。生活的面貌也仿佛隐隐看到背后的一些意义和灵动。

于是乎就去图书馆里看书,看计算机类的书,哲学类的书,也会偶尔看看小说,和传记。突
然明白,生命中重要的,不是做什么好,而是想去做什么。我要实现生命的意义,要有梦想,
一个真正意义上的梦想,那就是在技术浪潮下实现带给人们幸福的事情。当时看了 Eric S.
Raymond 的 《How To Become a Hacker》。我觉得这很适合我,做一个电脑技术人员,用
互联网技术去创造点什么。听我们老师说我们专业的学生从事本专业的人一个班级里最多
1\textasciitilde{}2个,我觉得继续浑浑噩噩的继续大学读下去也没有意思了,索性从现在开始去做自己喜
欢的事情,现在开始至少比等一年半开始要早一点。

\noindent\rule{\textwidth}{0.8pt}

\section{我的网上档案}
\label{sec:orge75bed5}

我大多数公开的东西都在GitHub上,但是笔记之类的都只在本地的Org Mode中,在搭建好平
台前,都无法在线浏览。

\begin{itemize}
\item \href{https://github.com/stardiviner/}{GitHub: stardiviner}
\item \href{https://stardiviner.github.io/}{Blog}
\item \href{https://stackexchange.com/users/366399/stardiviner}{my Stack Overflow profile}
\end{itemize}

\noindent\rule{\textwidth}{0.8pt}

\section{编程技术}
\label{sec:org3692fd6}

\subsection{操作系统:Linux}
\label{sec:org135f1d5}

从 \textit{[2011-10-10 Mon] } 年以来,我一直使用Linux系统作为我的日常系统,完全取代Windows
系统了。从最开始的Ubuntu用了2年左右,到后来转到Arch Linux下。后来一直使用Arch
Linux到现在。

\subsection{Clojure: 函数式基于 JVM 的 Lisp 语言}
\label{sec:org0cf2864}

\textit{[2017-08-16 Wed] } Clojure是我最熟悉的语言,平时都是用它。

\subsection{Emacs Lisp}
\label{sec:org36e95bc}

熟悉 Emacs Lisp 语言。会自己写一些小的插件和功能。

\subsection{了解一点其他语言:Python,Ruby,HTML,CSS,JS}
\label{sec:orgea8bc47}

\textit{[2015-03-16 Mon] } 很早以前学过一两个月的Python,Ruby。然后对于 HTML,CSS,JS 这些
语言有一点了解,但是不深入。

\subsection{熟悉正则表达式 regexp}
\label{sec:org37f5bb8}

\subsection{熟悉 Redis 和 MongoDB 的基础使用}
\label{sec:orga4ceb06}

\subsection{我自己用 Linux + Nginx + static site + Dynamic DNS 架过网站}
\label{sec:orge1d16b1}

\subsection{使用Git和开源社区的程序员协作,贡献patch}
\label{sec:orgc399d19}

\subsection{会用基本的 Docker 功能}
\label{sec:orga48f6c9}

\begin{itemize}
\item 了解 Docker Volume
\item 了解 Docker Compose
\end{itemize}

\subsection{脚本 Linux Shell Scripting}
\label{sec:org5be9123}

会用 Linux Shell 写一点脚本。

\subsection{参与开源社区的贡献}
\label{sec:org904c74c}

我平常浏览GitHub,看看有什么有意思的东西,关注动向,也会fork下一些插件,去提交几
个PR。平时在一些邮件列表里混,订阅了几个常看的,Emacs,Org Mode,Lisp,Clojure。
也会去一些社区论坛,比如 \href{https://emacs-china.org/}{Emacs China}, \href{https://clojureverse.org/}{Clojureverse}, 等等。


\noindent\rule{\textwidth}{0.8pt}

\section{编程工具:Emacs + Org-mode}
\label{sec:org12824eb}

很早以前在学Ubuntu Linux的时候,我学习使用的是Vim,用过大概有2年时间,后来因为
Org Mode的超级强大的功能的缘故,我转投Emacs了,一开始Emacs的按键我还不适应,但是
慢慢就好了,在一段时间的浸淫之后,我开始慢慢学习使用Org Mode,到现在,我已经为
Org Mode提供了好几个patch和feature了。也写了几个小功能的Emacs插件。我现在已经能
为我自己想要的功能Hack了。

现在我已经大多数工作都在Emacs和Org Mode下完成了:

\begin{itemize}
\item 写代码:我的主要语言是Clojure,在Emacs下使用CIDER,是我感觉IDE体验最好的工具。
Python有Elpy和lsp-python。也能满足大部分需求。而且Org Mode还支持 Literate
Programming paradigm,这是我在业余时间用的最多的。
\item 用Org Mode记笔记:我将所有笔记都记录在Org里了,我还自己谢了一个工具来快速准确
的搜索我想要的内容。我还打算将Org的内容host为搜索和浏览形式。这样手机上也能快
速搜索浏览了。
\item 任务管理:Org Mode有一个Agenda功能,用它管理任务,会更贴心。
\item 收发邮件
\item 和其他程序员在IRC上
\item 用Org Mode写博客
\end{itemize}

\subsection{contributed commits on Org Mode}
\label{sec:org2050506}

\begin{minted}[escapeinside=||,frame=lines,linenos=true,escapeinside=$$,mathescape=true,texcomments=true,numbersep=5pt,framesep=2mm,breaklines=true]{shell}
git log --author=stardiviner
\end{minted}


\noindent\rule{\textwidth}{0.8pt}

\section{关于技术学习规划}
\label{sec:orgdd766bc}

\begin{itemize}
\item 现在想加深爬虫技术的学习。所以要学习一些JavaScript,反爬虫和反反爬虫技术。
\item 深入学习并发和并行编程,函数式编程。
\item 了解一些数据分析的技术,Clojure上我已经开始关注一些库的使用。掌握统计分析的知识。
\end{itemize}

\noindent\rule{\textwidth}{0.8pt}

\section{其他技能}
\label{sec:org53b727c}

\subsection{英语四级,良好的英语阅读能力}
\label{sec:org0aa5302}

大学的时候通过了国家英语四级考试。

\subsection{Windows 桌面系统下}
\label{sec:org64c9278}

\begin{itemize}
\item 重装系统
\item 安装 pojie 软件
\item 会Windows下的高级设置,Windows下很多东西和Linux下都是很相似的。只是形式不一样。
\end{itemize}

\section{结尾语}
\label{sec:org854376f}

此文档由Org Mode嵌入\LaTeX{},导出为PDF文件。
\end{document}